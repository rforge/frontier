\documentclass[10pt]{article}

\usepackage{url}

\renewcommand{\title}[1]{\begin{center}{\bf \LARGE #1}\end{center}}
\newcommand{\affiliations}{\footnotesize}
\newcommand{\keywords}{\paragraph{Keywords:}}

\setlength{\oddsidemargin}{0cm} \setlength{\evensidemargin}{0cm}
\setlength{\textwidth}{16.5cm} \setlength{\topmargin}{-1cm}
\setlength{\textheight}{24.5cm}

\begin{document}
\pagestyle{empty}

\title{Efficiency Analysis in \textsf{R} using Parametric,\\[3mm]
Semiparametric, and Nonparametric Methods}

\begin{center}
  {\bf Arne Henningsen$^{1,2,*}$, Subal Kumbhakar$^{3}$}
\end{center}

\begin{affiliations}
1. Department of Agricultural Economics, University of Kiel (Germany)\par
2. Institute of Food and Resource Economics, University of Copenhagen (Denmark)\par
3. Department of Economics, State University of New York at Binghamton (USA)\par
* Contact author: arne.henningsen@gmail.com
\end{affiliations}

\keywords Efficiency, Productivity, Parametric, Semiparametric, Nonparametric

\vskip 0.8cm

Efficiency and productivity analysis is a major field
in applied production economics.
It is generally dominated by two methods:
the parametric Stochastic Frontier Analysis (SFA) and
the nonparametric and deterministic Data Envelopment Analysis (DEA).
The SFA can be done in \textsf{R} with the \textbf{frontier}
package~\cite{r-frontier-0.9}
and the DEA might be done with the \textbf{FEAR}%
\footnote{%
Please note that the non-academic use of the \textbf{FEAR} package
is restricted and that this closed-source software
is available as binary package for MS-Windows only.%
}
package~\cite{wilson08}.
The SFA approach contains a stochastic error term and hence,
is suitable even if there is some ``noise'' in the data.
However, this parametric approach requires the specification
of an explicit functional form,
although the functional form cannot be derived from theory.
Selecting a wrong functional form may lead to severely biased
estimation results.
If the data set includes production units
with rather different technologies,
even flexible functional forms
cannot model their production technologies adequately
and hence, the parametric SFA is inappropriate.
In contrast, the nonparametric and deterministic DEA approach does not require
the specification of a functional form,
but it does not include a stochastic component.
Hence, the DEA is not suitable in case of ``noisy'' data.

In many real world applications, the data are noisy \emph{and}
production units have rather different technologies (in parametric sense)
so that a stochastic and nonparametric approach is required
and neither the SFA nor the DEA is appropriate.
In cases like this,
a semiparametric SFA~\cite{fan96}
is appropriate,
because it allows for statistical ``noise''
and does not require the specification of a functional form
for production technologies.
In a first step, a nonparametric production function is estimated
and in a second step the residuals of the first step
are used to estimate inefficiencies.
Although in many empirical applications this approach seems to be
more appropriate than the SFA and DEA,
it has not been used much in applied studies,
probably because of nonavailability of user-friendly software.
However, several soft-ware packages for nonparametric econometrics
have become available in recent years.
For instance, the powerful and feature-rich \textbf{np} package~\cite{hayfield08}
can be used in the first step to estimate the nonparametric production function
and the \textbf{frontier} package~\cite{r-frontier-0.9}
can be used in the second step to estimate the technical efficiencies.

We will demonstrate how the three approaches (SFA, DEA, and semiparametric SFA)
can be used for applied efficiency analysis in \textbf{R}
and we compare the results obtained from all three approaches.

%\nocite{ref1,ref2}
\bibliographystyle{amsplain}
\bibliography{agrarpol}

\end{document}
